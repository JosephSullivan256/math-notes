\documentclass[class=report, crop=false]{standalone}
\usepackage{../../mystyle}

\begin{document}

\section{Introduction}

Here is a list of group theory topics that I hope to expand upon. For the time being, I will pick out interesting theorems and give proofs when I have time. I do hope to make something comprehensive at some point, however.
\begin{enumerate}
    \item Basic Definitions (Group, Subgroup, Homomorphism, Group Action)
    \item Important Subgroups (Centralizers, Normalizers, Stabilizers, Kernels)
    \item Generated Subgroups (Cyclic Groups, Characterization of Generated Subgroups)
    \item Cosets (Definition, Langrange's Theorem, Quotient Group)
    \item Isomorphism theorems
    \item Composition Series
    \item Transpositions and Alternating Group
    \item Group Actions (Stabilizers and Orbits, Left Multiplication and Cayley's Theorem, Conjugation)
    \item Methods of Finite Groups (Automorphisms, Sylow Theorems, Symplicity of $A_n$)
    \item Products (Direct Product, FTFGAG, Semidirect Products)
    \item Extra Group Theory Topics
\end{enumerate}

Before we jumping in, however, I will begin this chapter by cramming useful definitions for reference. There are just too many terms to remember, and I like having them all in one place.


\newpage

\section{Definitions}


\begin{boxedenv}\begin{definition}
    A \textbf{group} is a pair $(G,\cdot)$, where $G$ is a set of elements and $\cdot: G\times G\to G$ is a binary operation that (1) is associative, (2) has identity $1$ satisfying $1\cdot g = g\cdot 1 = g$ for all $g\in G$, and (3) has inverses $g^{-1}$ for each $g\in G$ satisfying $gg^{-1}=g^{-1}g=1$.
\end{definition}\end{boxedenv}

\begin{boxedenv}\begin{definition}
    A \textbf{subgroup} of a group $(G,\cdot)$ is a subset $H\subseteq G$ paired with the restricted binary operation $\cdot' = \left. \cdot \right|_{H\times H}$ such that $(H,\cdot')$ is a group. In that case, we write $H\le G$.
\end{definition}\end{boxedenv}

\begin{boxedenv}\begin{definition}
    Let $(G,\cdot_G),(H,\cdot_H)$ be groups. A \textbf{group homomorphism} is a function $\phi: G\to H$ such that
    \[\phi(g_1\cdot_G g_2) = \phi(g_1)\cdot_H \phi(g_2)\]
    for all $g_1,g_2\in G$. The \textbf{kernel} of $\phi$ is defined as
    \[\ker{\phi}=\{g\in G: \phi(g)=1_H\},\]
    and the \textbf{image} of $\phi$ is $\phi(G)=\{\phi(g)\in H: g\in G\}$.
\end{definition}\end{boxedenv}




\newpage

\noindent From now on, let $(G,\cdot)$ be a group.
\bigskip

\begin{boxedenv}\begin{definition}
    Let $X$ be an arbitrary set. A function $\mu: G\times A \to A$ is called a \textbf{group action} if it satisfies
    \[\mu(g_1 g_2, x) = \mu(g_1, \mu(g_2,x)), \qquad \mu(1,x) = x\]
    for all $g_1,g_2\in G$, $x\in X$. If the context is clear, we write instead $g\cdot x$ instead of $\mu(g,x)$. Then, the \textbf{orbit} of $x\in X$ is defined as
    \[G\cdot x = \{g\cdot x: x\in X\} \subseteq X,\]
    and the \textbf{stabilizer} of $x$ is defined as
    \[G_x = \{g\in G: g\cdot x = x\} \leq G.\]
\end{definition}\end{boxedenv}

\begin{boxedenv}\begin{definition}
    The \textbf{centralizer} of $x\in G$ is the subgroup
    \[C_G(x) = \{g\in G: gxg^{-1} = x\}.\]
    Equivalently, $C_G(a)$ is the stabilizer of $x\in G$ under the conjugation action. Then, for $A\subseteq G$,
    \[C_G(A) = \bigcap_{a\in A} C_G(a) = \{g\in G: gag^{-1} = a, \quad \forall a\in A\}.\]
    In other words, $C_G(A)$ is the subgroup of all $g\in G$ that commute with every element of $A$.
\end{definition}\end{boxedenv}

\begin{boxedenv}\begin{definition}
    The \textbf{center} of $G$ is defined as $Z(G)=C_G(G)$, i.e. the subgroup of all elements that commute with everything in $G$.
\end{definition}\end{boxedenv}

\begin{boxedenv}\begin{definition}
    The \textbf{normalizer} of $A\subseteq G$ is defined as
    \[N_G(A)=\{g\in G: gAg^{-1} = A\}.\]
\end{definition}\end{boxedenv}

\begin{boxedenv}\begin{definition}
    Let $A\subseteq G$. The \textbf{subgroup of $G$ generated by $A$} is defined as
    \[\<A\>=\bigcap_{\substack{A\subseteq H\\ H\leq G}} H.\]
\end{definition}\end{boxedenv}

\begin{boxedenv}\begin{definition}
    The \textbf{lattice of subgroups of $G$} is the set $L$ of subgroups of $G$, ordered by the subgroup relation $\le$. For two subgroups $A,B$, the meet is given by $A\land B=A\cap B$ and the join is given by $A\lor B=\<A\cup B\>$.
\end{definition}\end{boxedenv}



\end{document}
